% Numerical Computing II
% Homework 17
% Random Numbers
% February 20, 2010
% Problems 17.1,17.4,17.3 optional

\documentclass[11pt]{article}
\usepackage{listings}
\usepackage[fleqn]{amsmath}
\usepackage{amsfonts}
\usepackage{graphicx}
\begin{document}         
% Start your text
\newcommand{\makehomework}[2]%
{\begin{center}%
	\Huge #1\\%
	\Large #2\\%
	Marty Fuhry\\%
	\today%
\end{center}}
\makehomework{Numerical Computing II}{Homework 15: Zero Finders}

\lstset{language=Matlab,numbers=left,frame=single,breaklines=true,morecomment=[l]{//}}
\section*{Exercise 15.4}

\lstinputlisting{problem_15_4.m}
We define the functions $f(x)$ and $f'(x)$ as the error function and the error function's derivative. 
This code produces a very fast approximation for "nice" initial values of x. The initial values of
x are of crucial importance because the error function $f(x)$ returns 1 for all values of x greater 
than 3.3, which causes trouble for Newton's Method, as Newton's Method produces recursive values of
$x_{k}$ for when $f(x_{k+1}) = f(x_k)$, which happens frequently in this function.

For this function, we receive a zero after 6 iterations of the while loop with initial iterate of 2.

\section*{Exersize 15.5}
We set $a \in \mathbb{R}$. Then to find $a^{1/3}$ using Newton's Method, we can solve $x = a^{1/3}$ or, equivalently, $x^3 - a = 0$. So, we simply use
\begin{flalign}
    f(x) &= a - x^3\\
    f'(x) &= -3x^2
\end{flalign}

To use Newton's Method, we must specify an initial iterate, $b$, such that $x_1 = b$ and $x_2 = x_1 + f(b)/f'(b)$. We need to choose $b$ such that:
\begin{flalign*}
    x_2 &\neq x_1\\
    x_2 &\neq b.
\end{flalign*}

That is, we want to choose a $b$ which will not recursively return itself when given to Newton's Method:
\begin{flalign*}
    x_2 &\neq b - f(b)/f'(b)\\
    x_2 &\neq b - {a - b^3}/{-3b^2}\\
    b   &\neq \sqrt[3]{a}.
\end{flalign*}

The other conditions are simply that $b \neq 0$, as that would cause a division by zero to occur in our first iteration.

Our method can use a few termination conditions. First, we can't allow a division by zero, so stop our function when $x_i = 0$. Next, we realize that $x_i = \sqrt[3]{a}$ would cause our method to recurse, so we must stop at that condition. 
\section*{Exersize 15.6}
We need to find $x$ such that 
\begin{align}
    \label{15-6}
    \sigma x^3 = (1 + x)^2. 
\end{align}
That is, solve the function
\begin{align}
    f(x) = (1 + x)^2 - \sigma x^3.
\end{align}

\pagebreak
\lstinputlisting{problem_15_6.m}
This is quite easy to do with Newton's Method, and we receive the zero value of $x$ after 22 iterations of $x = 8.9651*10^{-04}$. This is equivalent to saying that \ref{15-6} reaches equality when the ratio of $m/M = x = 8.9651*10^{-04}$. Then, using this ratio, we can solve for $m$, the mass of Jupitor by using:
\begin{flalign*}
    m/M &= x\\
    \implies m &= x * M\\
    m &= 8.9651*10^{-04}* 1.989*10^{30}\\
      &= 1.7832*10^{27},
\end{flalign*}
which is a very close approximation of the actual mass of Jupiter, which is $1.8987*10^{27}$ kilograms.


% Stop your text
\end{document}
